\documentclass[a4paper]{article}

\usepackage[utf8x]{inputenc}    
\usepackage[T1]{fontenc}

\usepackage[francais,bloc,completemulti]{automultiplechoice}    
\begin{document}

\exemplaire{10}{    

%%% debut de l'en-tête des copies :    

\noindent{\bf QCM  \hfill TEST}

\vspace*{.5cm}
\begin{minipage}{.4\linewidth}
\centering\large\bf Test\\ Examen du 01/01/2008\end{minipage}
\champnom{\fbox{    
                \begin{minipage}{.5\linewidth}
                  Nom et prénom :

                  \vspace*{.5cm}\dotfill
                  \vspace*{1mm}
                \end{minipage}
         }}

\begin{center}\em
Durée : 10 minutes.

  Aucun document n'est autorisé.
  L'usage de la calculatrice est interdit.

  Les questions faisant apparaître le symbole \multiSymbole{} peuvent
  présenter zéro, une ou plusieurs bonnes réponses. Les autres ont
  une unique bonne réponse.

  Des points négatifs pourront être affectés à de \emph{très
    mauvaises} réponses.
\end{center}
\vspace{1ex}

%%% fin de l'en-tête

\begin{question}{prez}    
  Parmi les personnalités suivantes, laquelle a été présidente de la république française~?
  \begin{reponses}
    \bonne{René Coty}
    \mauvaise{Alain Prost}
    \mauvaise{Marcel Proust}
    \mauvaise{Claude Monet}
  \end{reponses}
\end{question}

\begin{questionmult}{pref}    
  Parmi les villes suivantes, lesquelles sont des préfectures~?
  \begin{reponses}
    \bonne{Poitiers}
    \mauvaise{Sainte-Menehould}
    \bonne{Avignon}
  \end{reponses}
\end{questionmult}

\clearpage    

}   

\end{document}
