\documentclass[a4paper]{article}

\usepackage[utf8x]{inputenc}
\usepackage[T1]{fontenc}

\usepackage[francais]{automultiplechoice}
\usepackage{multicol}

\begin{document}

\AMCtext{none}{Aucune des réponses ci-dessus}

%%% préparation des groupes

\element{Defaut}{
  \begin{questionmult}{Le noyau e__914}
    Le noyau et l'électron d'un atome d'hydrogène sont séparés par :
    \begin{multicols}{2}
    \begin{reponses}
      \mauvaise{De l'eau}
      \bonne{Du vide}
      \mauvaise{De l'air}
      \mauvaise{Nouvelle réponse}
    \end{reponses}
    \end{multicols}
  \end{questionmult}
}

\element{Defaut}{
  \begin{question}{L'etoile l__468}
    L'étoile la plus proche de la Terre est :
    \begin{multicols}{2}
    \begin{reponses}
      \bonne{Le Soleil.}
      \mauvaise{La Lune.}
      \mauvaise{Mars.}
    \end{reponses}
    \end{multicols}
  \end{question}
}

\element{Defaut}{
  \begin{question}{Dans la mo__294}
    Dans la molécule de dihydrogène \[H_{2}\], la distance entre les noyaux d'hydrogène est 0,3 nm. L'ordre de grandeur de cette distance est :
    \begin{multicols}{2}
    \begin{reponses}
      \bonne{\[10^{-10} m\].}
      \mauvaise{\[10^{-8}\: m\].}
      \mauvaise{\[0,3*10^{-9}\: m\].}
    \end{reponses}
    \end{multicols}
  \end{question}
}

\element{Defaut}{
  \begin{question}{La valeur __229}
    La valeur de la vitesse de la lumière dans le vide est envion :
    \begin{multicols}{2}
    \begin{reponses}
      \mauvaise{\[3,00*10^{-8}\: m.s^{-1}\].}
      \bonne{\[3,00*10^{8}\: m.s^{-1}\].}
      \mauvaise{\[3,00*10^{8}\: km.s^{-1}\].}
    \end{reponses}
    \end{multicols}
  \end{question}
}

\element{Defaut}{
  \begin{question}{Un spectre__805}
    Un spectre lumineux est le résultat de :
    \begin{multicols}{2}
    \begin{reponses}
      \bonne{La décomposition de la lumière.}
      \mauvaise{La recomposition de la lumière.}
      \mauvaise{La propagation de la lumière.}
    \end{reponses}
    \end{multicols}
  \end{question}
}

\element{Defaut}{
  \begin{questionmult}{Un gaz exc__677}
    Un gaz excité sous faible pression émet de la lumière. Le spectre de cette lumière est :
    \begin{multicols}{2}
    \begin{reponses}
      \bonne{Un spectre de raies.}
      \mauvaise{Un spectre continu.}
      \bonne{Un spectre d'emission.}
    \end{reponses}
    \end{multicols}
  \end{questionmult}
}

\element{Defaut}{
  \begin{questionmult}{Une entite__181}
    Une entité chimique gazeuse, sous faible pression, est éclairée par de la lumière blanche. Le spectre de la lumière ayant traversé ce gaz est :
    \begin{multicols}{2}
    \begin{reponses}
      \bonne{Caractéristique de cette entité chimique.}
      \bonne{Constitué de raies noires sur fond coloré.}
      \mauvaise{Appelé spectre d'émission.}
    \end{reponses}
    \end{multicols}
  \end{questionmult}
}

\element{Defaut}{
  \begin{question}{Une etoile__779}
    Une étoile bleue est:
    \begin{multicols}{2}
    \begin{reponses}
      \bonne{Plus chaude qu'une étoile rouge.}
      \mauvaise{Plus froide q'une étoile rouge.}
      \mauvaise{Plus proche de l'observateur qu'une étoile rouge.}
    \end{reponses}
    \end{multicols}
  \end{question}
}

\element{Defaut}{
  \begin{questionmult}{Lors d'une__953}
    Lors d'une réfraction, la lumière change :
    \begin{multicols}{2}
    \begin{reponses}
      \mauvaise{De couleur.}
      \bonne{De direction de propagation.}
      \bonne{De milieu de propagation.}
    \end{reponses}
    \end{multicols}
  \end{questionmult}
}

\element{Defaut}{
  \begin{question}{Le phenome__688}
    Le phénomène de réfraction peut se produire si :
    \begin{multicols}{2}
    \begin{reponses}
      \mauvaise{Les milieux traversés sont transparents et ont des indices de réfraction égaux.}
      \bonne{Les milieux traversés sont transparents et ont des indices de réfraction différents.}
      \mauvaise{L'angle d'incidence est nul.}
    \end{reponses}
    \end{multicols}
  \end{question}
}

\element{Defaut}{
  \begin{question}{Un prisme __516}
    Un prisme disperse de la lumière blanche car son indice de réfraction dépend de :
    \begin{multicols}{2}
    \begin{reponses}
      \bonne{La longueur d'onde des radiations qui le traversent.}
      \mauvaise{L'angle d'incidence.}
      \mauvaise{Sa forme géométrique.}
    \end{reponses}
    \end{multicols}
  \end{question}
}

\element{Defaut}{
  \begin{questionmult}{La lumiere__827}
    La lumière qui nous parvient d'une étoile s'est propagée dans le vide avant de pénétrer dans l'atmosphère terrestre, dont l'indice de réfraction varie en fonction de l'altitude.
    \begin{multicols}{2}
    \begin{reponses}
      \mauvaise{Cette lumière se propage en ligne droite depuis l'étoile jusqu'à l'observateur.}
      \bonne{Cette lumière se propage en ligne droite dans le vide.}
      \bonne{Cette lumière est réfractée par l'atmosphère terrestre.}
    \end{reponses}
    \end{multicols}
  \end{questionmult}
}

\element{Defaut}{
  \begin{question}{Le noyau d__878}
    Le noyau d'un atome de fer comportant 26 protons et 30 neutrons est caractérisé par :
    \begin{multicols}{2}
    \begin{reponses}
      \mauvaise{A = 26 et Z = 30.}
      \mauvaise{A = 56 et Z = 30.}
      \bonne{A = 56 et Z = 26.}
    \end{reponses}
    \end{multicols}
  \end{question}
}

\element{Defaut}{
  \begin{questionmult}{Un atome d__131}
    Un atome de cobalt Co qui a perdu 3 électrons devient :
    \begin{multicols}{2}
    \begin{reponses}
      \bonne{un ion \[Co^{3+}\].}
      \mauvaise{Un ion \[Co^{3-}\].}
      \bonne{Un cation.}
    \end{reponses}
    \end{multicols}
  \end{questionmult}
}

\element{Defaut}{
  \begin{question}{"Les atome__235}
    "Les atomes \[_{1}^{2}\textrm{H}\]
 et \[_{2}^{3}\textrm{He}\]
 :
    \begin{multicols}{2}
    \begin{reponses}
      \mauvaise{Sont isotopes.}
      \mauvaise{Appartiennent au même élément chimique.}
      \bonne{Ont le même nombre de neutrons.}
    \end{reponses}
    \end{multicols}
  \end{question}
}

\element{Defaut}{
  \begin{question}{Un atome p__61}
    Un atome possédant 9 électrons a :
    \begin{multicols}{2}
    \begin{reponses}
      \mauvaise{1 électron externe.}
      \bonne{7 électrons externes.}
      \mauvaise{9 électrons externes.}
    \end{reponses}
    \end{multicols}
  \end{question}
}

\element{Defaut}{
  \begin{question}{"Un atome __654}
    "Un atome de formule électronique \[(K)^{2}(L)^{7}\]
 :"
    \begin{multicols}{2}
    \begin{reponses}
      \bonne{Tend à respecter la règle de l'octet.}
      \mauvaise{Tend à respecter la règle du duet.}
      \mauvaise{Ne forme pas d'ions.}
    \end{reponses}
    \end{multicols}
  \end{question}
}

\element{Defaut}{
  \begin{question}{Pour obten__629}
    Pour obtenir la formule électronique d'un gaz noble, un atome de magnésium Mg de formule électronique \[(K)^{2}(L)^{8}(M)^{2}\]
 peut former :
    \begin{multicols}{2}
    \begin{reponses}
      \bonne{Le cation \[Mg^{2+}\].
}
      \mauvaise{L'atome Ne.}
      \mauvaise{L'anion \[Mg^{6-}\].}
    \end{reponses}
    \end{multicols}
  \end{question}
}

\element{Defaut}{
  \begin{question}{Un signal __413}
    Un signal périodique a obligatoirement la (les) propriété(s) suivante(s) :
    \begin{multicols}{2}
    \begin{reponses}
      \mauvaise{Le motif est symétrique par rapport à l'axe des abscisses.}
      \bonne{Le motif se répète à des intervalles de temps de même durée.}
      \mauvaise{Le motif présente alternativement des valeurs positives et négatives.}
    \end{reponses}
    \end{multicols}
  \end{question}
}

\element{Defaut}{
  \begin{questionmult}{Dans une f__552}
    Dans une fibre optique de fibroscope, la lumière reste canalisée à l'intérieur de la fibre grâce :
    \begin{multicols}{2}
    \begin{reponses}
      \mauvaise{à la réfraction.}
      \mauvaise{à la réflexion limite.}
      \bonne{à la réflexion totale.}
    \end{reponses}
    \end{multicols}
  \end{questionmult}
}

%%% fabrication des copies

\exemplaire{10}{

%%% debut de l'en-tête des copies :

\noindent{\bf QCM \hfill TEST}

\vspace*{.5cm}
\begin{minipage}{.4\linewidth}
  \centering\large\bf QCM éclair !
\end{minipage}
\champnom{\fbox{\begin{minipage}{.5\linewidth}
Nom et prénom

\vspace*{.5cm}\dotfill
\vspace*{1mm}
\end{minipage}}}

%%% fin de l'en-tête

\begin{center}
  \hrule\vspace{2mm}
  \bf\Large Défaut
  \vspace{2mm}\hrule
\end{center}

\melangegroupe{Defaut}
\restituegroupe{Defaut}

\clearpage

}

\end{document}
